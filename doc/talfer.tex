\documentclass[11pt,a4paper]{article}

%% Preamble
\usepackage[utf8]{inputenc}
\usepackage{amsmath}
\usepackage{amsfonts}
\usepackage{amssymb}
\usepackage{url}
\usepackage{listings}
\usepackage{multicol}

\lstset{%
  showstringspaces=false,
  breaklines=true,%
  frame=single,%
  language=C,%
  basicstyle=\small,%
  numbers=left,
  numberstyle=\tiny,
  basicstyle=\ttfamily\fontsize{10}{10}\selectfont,
  showstringspaces=false,
  showstringspaces=false}

\renewcommand{\arraystretch}{1.5}

\begin{document}


%%%%%%%%%%%%%%%%%%%%%%%%%%%%%%%%%%%%%%%%%%%%%%%%%%%%%%%%%%%%%%%%%%%%%%%%
% TITLE
%%%%%%%%%%%%%%%%%%%%%%%%%%%%%%%%%%%%%%%%%%%%%%%%%%%%%%%%%%%%%%%%%%%%%%%%

\title{Talfer: A framework for interactive computational fluid simulation and
visualization}

\maketitle



%%%%%%%%%%%%%%%%%%%%%%%%%%%%%%%%%%%%%%%%%%%%%%%%%%%%%%%%%%%%%%%%%%%%%%%%
% INTRODUCTION
%%%%%%%%%%%%%%%%%%%%%%%%%%%%%%%%%%%%%%%%%%%%%%%%%%%%%%%%%%%%%%%%%%%%%%%%
\section{Introduction}

Talfer is a framework for interactive simulation and visualization of
steered computational fluid and rigid body simulations.
It is based on pipelining concept further explained in Sec.\,\ref{sec:pipeline}
and a modular communication infrastructure.

This document gives an introduction to the basic concepts of the framework and
describes the requirements of it.



%%%%%%%%%%%%%%%%%%%%%%%%%%%%%%%%%%%%%%%%%%%%%%%%%%%%%%%%%%%%%%%%%%%%%%%%
% Requirements \& Compilation
%%%%%%%%%%%%%%%%%%%%%%%%%%%%%%%%%%%%%%%%%%%%%%%%%%%%%%%%%%%%%%%%%%%%%%%%
\section{Requirements \& Compilation}

Talfer was developed to be compiled and executed on Ubuntu 12.04 LTS
\footnote{\url{http://www.ubuntu.com/download/desktop}}.

After downloading and installing the system, several packages are required.
Those can be installed using \textit{apt-get} from the command line.

\begin{lstlisting}[language=sh]
$ apt-get install libsdl-image1.2-dev libsdl1.2-dev libv4l-dev g++
\end{lstlisting}

For eclipse development environment, additionally install java jre
\begin{lstlisting}[language=sh]
$ apt-get install default-jre
\end{lstlisting}

After installation of the required packages, Talfer can be compiled with

\begin{lstlisting}[language=sh]
$ make
\end{lstlisting}

In case that you get any compilation errors, please send your advisors an
email. The program will be located in the build folder and can be run by

\begin{lstlisting}[language=sh]
$ ./build/fa_2013_release
\end{lstlisting}




%%%%%%%%%%%%%%%%%%%%%%%%%%%%%%%%%%%%%%%%%%%%%%%%%%%%%%%%%%%%%%%%%%%%%%%%
% Classes
%%%%%%%%%%%%%%%%%%%%%%%%%%%%%%%%%%%%%%%%%%%%%%%%%%%%%%%%%%%%%%%%%%%%%%%%

\section{Classes}

There are a bunch of existing classes which should be used by everyone to setup
the interfaces among the groups.
The following table gives an overview and short description of each class.

\noindent
\begin{tabular}{|l|l|}
	\hline
	CDataArray2D.hpp				&
		Data storage for 2D arrays							\\
	\hline
	CDataDrawingInformation.hpp		&
		Data storage for interactive drawing information	\\
	\hline
	CGlTexture.hpp					&
		Abstraction for OpenGL Textures						\\
	\hline
	CParameters.hpp					&
		Program and simulation parameters					\\
	\hline
	CPipelinePacket.hpp				&
		Pipeline packet capable of being forwarded via the pipeline	\\
	\hline
	CPipelineStage.hpp				&
		Pipeline stage providing interfaces for pipelining	\\
	\hline
	CSDLInterface.hpp				&
		SDL Interface for visualization and interactivity	\\
	\hline
	CStage\_ImageInput.hpp			&
		Pipeline stage for image input (single image from file)	\\
	\hline
	CStage\_ImageProcessing.hpp		&
		Pipeline stage for image processing filter			\\
	\hline
	CStage\_VideoInput.hpp			&
		Pipeline stage for video input, e.g.\,from webcam	\\
	\hline
	CStage\_VideoOutput.hpp			&
		Pipeline stage for video output				\\			
	\hline
	main.cpp						&
		main entry to setup pipelined scenarios		\\
	\hline
\end{tabular}


%%%%%%%%%%%%%%%%%%%%%%%%%%%%%%%%%%%%%%%%%%%%%%%%%%%%%%%%%%%%%%%%%%%%%%%%
% Pipeline
%%%%%%%%%%%%%%%%%%%%%%%%%%%%%%%%%%%%%%%%%%%%%%%%%%%%%%%%%%%%%%%%%%%%%%%%
\section{Pipeline}
\label{sec:pipeline}

The idea of a pipelining model is creating independent execution parts with
particular input-output specifications.
E.g.\,a webcam only provides images which are forwarded to the image filter.
After processing of the image filter, this information is further forwarded to
the simulation (not yet implemented) and then to the output for visualization.

All those pipeline stages are independent and thus can be independently
processed - e.g. Image filter and simulation computations in parallel.

%%%%%%%%%%%%%%%%%%%%%%%%%%%%%%%%%%%%%%%%%%%%%%%%%%%%%%%%%%%%%%%%%%%%%%%%
% Pipeline stage
%
\subsection{Pipeline stage}

We continue with an example given by the image processing filter
\textit{CStage\_ImageProcessing.hpp}.
For well-known interfaces, each new pipeline stage has to inherit the class
CPipelineStage:

\begin{lstlisting}
class CStage_ImageProcessing	:	public
	CPipelineStage
...
\end{lstlisting}

\noindent
For processing of the images, paramters are required to know which
computations to do, at least a single input storage is required as well as an
output storage to forward processes images to other classes:

\begin{lstlisting}
...
/**
 * global parameters
 */
CParameters &cParameters;

/**
 * input image
 */
CDataArray2D<unsigned char,3> input_cDataArray2D;

/**
 * processed image
 */
CDataArray2D<unsigned char,3> output_cDataArray2D;
...
\end{lstlisting}

\noindent
Since the parameters are shared with the other classes, they are
setup in the constructor:

\begin{lstlisting}
public:
/**
 * constructor
 */
CStage_ImageProcessing(CParameters &i_cParameters):
CPipelineStage("ImageProcessing"),
cParameters(i_cParameters)
{
}
\end{lstlisting}

\noindent
In case of an input sent via the pipeline of another stage such as the video
input, the method \textit{pipeline\_process\_input} is executed and has to be
implemented. This interface is particularly requested by the class
\textit{CPipelineStage}.

\begin{lstlisting}
void pipeline_process_input(
	CPipelinePacket &i_cPipelinePacket
)
{
...
\end{lstlisting}

\noindent
Since not all possible data types can be probably processed, we have to check
for compatible input packages and unpack the data to make it available with our
accessor class:
\begin{lstlisting}
// we are currently only able to process "unsigned char,3" data arrays.
if (i_cPipelinePacket.type_info_name != typeid(CDataArray2D<unsigned char,3>).name())
{
	std::cerr << "ERROR: Video Output is only able to process (char,3) arrays" << std::endl;
	exit(-1);
}

// unpack data
CDataArray2D<unsigned char,3> *input = i_cPipelinePacket.getPayload<CDataArray2D<unsigned char,3> >();
\end{lstlisting}

\noindent
After unpacking the data and processing the data with more details available in
the source code file itself, the output data array is pushed to the pipeline and
thus forwarded to the next pipeline stages:
\begin{lstlisting}
CPipelineStage::pipeline_push((CPipelinePacket&)output_cDataArray2D);
\end{lstlisting}

%%%%%%%%%%%%%%%%%%%%%%%%%%%%%%%%%%%%%%%%%%%%%%%%%%%%%%%%%%%%%%%%%%%%%%%%
% Pipeline setup
%
\subsection{Pipeline setup}

After programming several stages, their input and output has to be
connected after instantiation.
E.g.\,let us assume that we like to have an static input image with an image
filter and the possibility to draw into the image, this would lead to the
following pipeline:

\begin{lstlisting}
// static image input
CStage_ImageInput cStage_ImageInput(cParameters);
// video output
CStage_VideoOutput cStage_VideoOutput(cParameters);

// PIPELINE CONNECTIONS
// forward image to video output
cStage_ImageInput.connectOutput(cStage_VideoOutput);
// forward mouse movements to image input
cStage_VideoOutput.connectOutput(cStage_ImageInput);


// initial push of static image
cStage_ImageInput.pipeline_push();

// main loop
while (!cParameters.exit)
{
	// trigger image input to do something
	cStage_VideoOutput.main_loop_callback();
}
\end{lstlisting}

\noindent
For our pipeline concept, only the outputs have to be connected.
The initial push for the image input is required to initially forward the static
image to the video output.

The main loop is required to e.g.\,check for user input, to draw updates for the
visualization and to run a simulation timestep.


\section{Interaction}

Several keystrokes currently exist updating some parameters in the class
\textit{CParameters}.
Note that all pipeline stages get a reference to this class during
initialization.

Since the Video output is closely connected to the input system, all
input keystrokes which are not directly processed are forwarded to the method
key\_down of the parameter class:

\begin{lstlisting}
/**
 * return bool if processed
 */
bool key_down(char i_key)
{
	switch(i_key)
	{
	case SDLK_j:
		stage_imageprocessing_filter_id--;
		std::cout << "Using filter id " << stage_imageprocessing_filter_id << std::endl;
		return true;

	case SDLK_k:
		stage_imageprocessing_filter_id++;
		std::cout << "Using filter id " << stage_imageprocessing_filter_id << std::endl;
		return true;
\end{lstlisting}

\noindent
This allows the modification of the parameters during the programs runtime.
So far the following keystrokes are defined:

\noindent
\begin{tabular}{|c|l|}
	\hline
	\multicolumn{2}{|l|}{\textbf{General}}	\\
	\hline
	q & quit program	\\
	\hline
	\hline

	\multicolumn{2}{|l|}{\textbf{1: image Processing}}	\\
	\hline
	j,k & decrease / increase filter id\\
	\hline
	g,t & decrease/increase threshold value\\
	\hline
\end{tabular}


\section{Programn start}

Several program parameters currently exist and are also processed in
CParameters:

\noindent
\begin{tabular}{|c|l|}
	\hline
	\multicolumn{2}{|l|}{\textbf{General}}	\\
	\hline
	p	& pipeline id to use (see main.cpp)	\\
	\hline
	v	& verbosity level	\\
	\hline
	\hline

	\multicolumn{2}{|l|}{\textbf{0: image/videoinput}}	\\
	\hline
	d	& video device string to use\\
	\hline
	w	& request this width for video input\\
	\hline
	h	& request this height for video input\\
	\hline
	i	& path to input image to use\\
	\hline
	\hline

	\multicolumn{2}{|l|}{\textbf{3: fluid simulation LBM}}	\\
	\hline
	v	& switch between flag field and velocity output	\\
	\hline
	\hline
	
	\multicolumn{2}{|l|}{\textbf{4: parallelization}}	\\
	\hline
	n	& number of threads to use\\
	\hline
	\hline

\end{tabular}

\section{Group 0: Management}
\input{group_0_management.tex}


\section{Group 1: Input devices}
More information for installation of the Kinect drivers for linux is available in kinect.txt



\section{Group 2: Image processing}

\subsection{Input}
The goal of the image processing is to identify the game field out of a webcam input. This is done by applying several filters to the given image and finally creating a flagfield which only contains the flags 0, 1, 2, 3, and 4 representing the flow area, obstacles, input area, output area, and target area (see \ref{sec:imgprocOutput}). The flagfield will then be passed to the other stages of the pipeline. \\

\noindent The algorithms and parameters used for the image processing are implemented in the following three files:
\begin{itemize}
	\item CStage\_ImageProcessing.hpp: contains the filters used to identify the game field
	\item imageops.c: basic image operations that are used by the filters. These operations do not use any objects from the CStageImageProcessing file and are basic C methods. 
	\item image\_parameters.h: in this file all fixed parameters are declared that are used in the image processing. If the image processing does not work as aspected, first try to fix the parameters (especially the brightness values for the input, output, target area detection) in this file to achieve better results.
\end{itemize}

\noindent Before applying several filters, the given image is transformed into an internal flat buffer format using the to\_internal\_buffer() method. The given input array representing the current frame is a two-dimensional array and has the format [R,G,B,R,G,B,R,G,B,R,G,B,...]. Within the internal buffer the three color channels are split up which results into the buffer format [R,R,R,R,R,...,G,G,G,G,G,...,B,B,B,B,...] that allows a easier implementation of the used filters. After processing the frame, the internal flat buffer format is retransformed to the input array format. \\

\noindent The problems we faced implementing the image processing are due to noise and the correct detection of the colors - especially in the borders of the input, output, and target area. However, using the following filters, we got very good results:

\subsection{Used Filters}
In the following we will give a short introduction to the filters we used to create the flagfield:
\begin{itemize}
	\item[1.] \textbf{Difference of Gauss Filter (DoG), apply\_filter\_DoG():} This filter will create two images - a fine blurred and a rough blurred image - using an approximation of a Gaussian Blur filter. Then the difference of these blurred images is calculated which gives you a rough approximation of a bandpass filter. The reason for applying this filter is to get rid of all fine details and noise that are not needed to identify the game field. 

	\item[2.] \textbf{Threshold filter with hysteresis, apply\_filter\_hysteresis():}  This filter applies thresholds to the DoG response to detect Objects. To reduce flickering and make the output more stable, hysteresis in time is used: when a pixel was detected as object in the last frame, a lower threshold is applied; else a higher. 
	
	\item[3.] \textbf{HSV color space filter, apply\_filter\_flagfield():} The colors of the frame are interpreted as HSV color values now. Using the hue and the saturation helps to identify the different areas of the game field. In the current implementation the brightness values of the colors are also used to distinguish between red, blue, and green from black. Whenever the colors red, blue, or green are identified as black colors, you might need to adapt the brightness values to the current lighting conditions.

	\item[4.] \textbf{flagfield post processing, apply\_flagfield\_postprocess():} The flagfield at this stage still contains a lot of black pixels around the red input, the blue output, and the green target area. The flagfield post process filter iterates a couple of times over the image and replaces the color of every black pixel with a red, blue, or green color value if the number of colored pixels around the black pixel exceed a given threshold. The different threshold for green, red, or blue pixels can be found (like all other hardcoded parameters) in the image\_parameters.h file.

	\item[5.] \textbf{Smoothing edges, apply\_flagfield\_pixelfix():} The Navier Stokes equations cannot handle single pixel lines. This filter makes sure that there are no longer single pixels that are salient or missing in a line.

\end{itemize}

\subsection{Unused filters and FFT}
\noindent There are still some filters and basic image functions in the CStage\_ImageProcessing and image\_parameters.h file that are currently not used by our implementation because of their runtime. Since they worked well, we decided to keep them in our code to allow further testing and development. 
\begin{itemize}
	\item \textbf{Diffusion Filter}: This filter will also blur the image but will also keep sharp edges. This feature distinguishes it from the Gaussian Blur filter. However, our implementation of this filter is very slow so we decided to use a Difference of Gauss filter instead which is way faster and also gives satisfying results.
	\item \textbf{FFT}: All the DoG operations could be performed in one step in the frequency domain, using arbitrarily complex convolution kernels. To do this, the traditional filter implementation ist first applied to an image with a single white pixel to get the filters impulse response which is equivalent to the used kernels. Then, the kernel is fourier-transformed and stored for usage. Now, for every frame, the image is fourier-transformed, then multiplied pointwise with the transformed kernel and then transformed back. While the implementation is basically working, it turned out to be not really faster than the normal implementation and does not fit well to the filtering model. Therefore, it is currently disabled, but it can be used by enabling the \verb-USE_FFT- flag. The make target \verb-fft- does this and also performs the requiered linking to the used \verb-fftw3- library.
\end{itemize}

\subsection{Change Detection}
There are two kinds of change detection in the image processing stage: the first detects whether the camera image has not changed at all. When there are only few changes to the input images, the old output is send again to save computation work in the image processing and in the following simulation stages.\\

\noindent The second detects whether there are many changes in the camera image. This is the case when the user is modifying the image right now, and there are hands in the image. Frames with many changes in between are also discarded.

\subsection{Output}
\label{sec:imgprocOutput}
\noindent After applying the filters above a frame is pushed to the pipeline which is used again by other pipeline stages. The values in the flagfield can be interpret in the following way:\newline
	
	\begin{tabular}{c|l} 
		\textbf{flag number} & \textbf{interpretation}\\
		\hline
		0 & flow area\\
		1 & black, obstacles\\
		2 & red, input area \\
		3 & blue, output area \\
		4 & green, target area
	\end{tabular}
	
	



\section{Group 3a: Lattice Boltzmann simulation}
%!TEX root = /Users/codemonk/uni/fau/Ferienakademie/svn/trunk/doc/talfer.tex

There are two versions of LBM in the trunk: one which is parallelized with Threaded Building Blocks (see \textit{src/CStage\_FluidSimulationLBM\_TBB.hpp}) and one with OpenMP (see \textit{src/CStage\_FluidSimulationLBM.hpp}). The TBB version is a little more performant, but has a higher complexity due to the required class structure.

Both versions have the following structure:
\begin{itemize}
    \item This is an D2Q9 implementation (this means two dimensional and with nine directional probabilitydistributions per cell/pixel)
    \item Output is written to:
    \begin{itemize}
        \item \textit{output\_pressure}, of type \textit{CDataArray2D<float,1>} (this is actually the density and would need to be multiplied by the gas constant and temperature) 
        \item \textit{output\_velocity}, of type \textit{CDataArray2D<float,2>}, there the first component is the velocity in horizontal direction and the second in vertical direction.
    \end{itemize} 
    \item Input: \textit{input\_cDataArray2D}, of type \textit{CDataArray2D<unsigned char,1>} with the following flag interpretation:
    \begin{itemize}
        \item 0: Fluid Cell
        \item 1: Obstacle cell
        \item 2: Inflow cell
        \item 3: Outflow cell
        \item 4: Fluid Cell (is used for target zone definition in Game)
    \end{itemize}
    \item All boundaries are considered periodic
    \item The simulation acts on \textit{simulationData\_BufferSrc} and \textit{simulationData\_BufferDst}, and swaps those after each time-step
    \item Two time-steps are performed per \textit{main\_loop\_callback}
\end{itemize}



\section{Group 3b: Navier-Stokes simulation}
\input{group_3b_navier_stokes_simulation.tex}


\section{Group 4: Parallelization}
\input{group_4_parallelization.tex}


\section{Group 5: Rigid body simulation}

\subsection{Class setup}
The rigid body part is made up of three classes. The Vector2 class provides vector operations. CParticle represents one rigid body with different properties like mass, position, velocity and a method to calculate the euler integration. The main loop of CStage\_RigidBodySimulation calls all needed methods. Movement of all particles, checks for collision and adds the force from the fluid field and the player.
\subsection{Integration}
\subsubsection{Input}
The simulation receives several packets via the talfer pipeline:
\begin{itemize}
\item Flag field (from VideoInput)
Used to look for inflows, outflows and obstacles to detect collisions.
\item Fluid velocity field (from LBM or NS)
Used to influence the rigid bodies depending on the velocity of the fluid.
\item Kinect/Keyboard force
Used to apply an additional force on the rigid bodies.
\end{itemize}
\subsubsection{Output}
We need to send an array of our rigid bodies to the Visualization to provide a visual representation. The array contains a std::vector of all particles with their current positions.
Therefore we use our pipeline packet class CDataParticleArray.
\subsubsection{Game integration}
For the gameplay we need to send information about the lifecycle of rigid bodies to the Game class.
\begin{itemize}
\item particle\_created(): A new particle was created.
\item particle\_bumped(): called on obstacle hits in order to play sound.
\item particle\_destroyed(): A particle vanished at an outflow. Additional information like mass, number of obstacle hits and the kind of obstacle are sent.
\end{itemize}
Different Game scenarios need differnt kinds of rigid bodies. Before a new rigid body is created the function Game::setNextParticle(CParticle* p) is called so that custom modifications can be made. For example setting different sprites (Kaiserschmarrn, penguins or blowfish) and masses.
\subsection{Euler Timsteps (Integration)}
\subsection{Collision Detection}
\subsection{Collision Resolution}
\subsubsection{Collision with rigid body}
\subsubsection{Collision with obstacles}
\subsection{Fluid force on particles}
name of formulae?
\subsection{Rotation}


\subsection{Useful parameters to tweak}
\begin{itemize}
\item area for force of fluid on rigid bodies
\item addForce of kinect
\item ...
\end{itemize}



\section{Group 6: Visualization}
\input{group_6_visualization.tex}

  
\end{document}
