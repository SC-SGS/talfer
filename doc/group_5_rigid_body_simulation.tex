
\subsection{Class setup}
The rigid body part is made up of three classes. The Vector2 class provides vector operations. CParticle represents one rigid body with different properties like mass, position, velocity and a method to calculate the euler integration. The main loop of CStage\_RigidBodySimulation calls all needed methods. Movement of all particles, checks for collision and adds the force from the fluid field and the player.
\subsection{Integration}
\subsubsection{Input}
The simulation receives several packets via the talfer pipeline:
\begin{itemize}
\item Flag field (from VideoInput)
Used to look for inflows, outflows and obstacles to detect collisions.
\item Fluid velocity field (from LBM or NS)
Used to influence the rigid bodies depending on the velocity of the fluid.
\item Kinect/Keyboard force
Used to apply an additional force on the rigid bodies.
\end{itemize}
\subsubsection{Output}
We need to send an array of our rigid bodies to the Visualization to provide a visual representation. The array contains a std::vector of all particles with their current positions.
Therefore we use our pipeline packet class CDataParticleArray.
\subsubsection{Game integration}
For the gameplay we need to send information about the lifecycle of rigid bodies to the Game class.
\begin{itemize}
\item particle\_created(): A new particle was created.
\item particle\_bumped(): called on obstacle hits in order to play sound.
\item particle\_destroyed(): A particle vanished at an outflow. Additional information like mass, number of obstacle hits and the kind of obstacle are sent.
\end{itemize}
Different Game scenarios need differnt kinds of rigid bodies. Before a new rigid body is created the function Game::setNextParticle(CParticle* p) is called so that custom modifications can be made. For example setting different sprites (Kaiserschmarrn, penguins or blowfish) and masses.
\subsection{Euler Timsteps (Integration)}
\subsection{Collision Detection}
\subsection{Collision Resolution}
\subsubsection{Collision with rigid body}
\subsubsection{Collision with obstacles}
\subsection{Fluid force on particles}
name of formulae?
\subsection{Rotation}


\subsection{Useful parameters to tweak}
\begin{itemize}
\item area for force of fluid on rigid bodies
\item addForce of kinect
\item ...
\end{itemize}
